\documentclass[sigconf,nonacm]{acmart}
%\documentclass[sigconf,nonacm,anonymous=true]{acmart}

%% Rights management information.  This information is sent to you
%% when you complete the rights form.  These commands have SAMPLE
%% values in them; it is your responsibility as an author to replace
%% the commands and values with those provided to you when you
%% complete the rights form.
%\setcopyright{acmlicensed}
%\copyrightyear{2018}
%\acmYear{2018}
%\acmDOI{XXXXXXX.XXXXXXX}

%% These commands are for a PROCEEDINGS abstract or paper.
\acmConference[COTB]{XVI Computer on the Beach}{April 02--05,
  2025}{Itajaí, SC}
%%
%%  Uncomment \acmBooktitle if the title of the proceedings is different
%%  from ``Proceedings of ...''!
%%
%%\acmBooktitle{Woodstock '18: ACM Symposium on Neural Gaze Detection,
%%  June 03--05, 2018, Woodstock, NY}
%\acmISBN{978-1-4503-XXXX-X/18/06}


\begin{document}

\title{Paper Title}

\author{Aluno 1}
\email{gabrielrazzolini@ufpr.br}
\orcid{XXXX-XXXX-XXXX}
\affiliation{%
  \institution{Departamento de Informática\\Universidade Federal do Paraná}
  \city{Curitiba}
  \state{Paraná}
  \country{Brasil}
}

\author{Aluno 2}
\email{?@ufpr.br}
\orcid{XXXX-XXXX-XXXX}
\affiliation{%
	\institution{Departamento de ...\\Universidade Federal do Paraná}
	\city{Curitiba}
	\state{Paraná}
	\country{Brasil}
}

%%
%% By default, the full list of authors will be used in the page
%% headers. Often, this list is too long, and will overlap
%% other information printed in the page headers. This command allows
%% the author to define a more concise list
%% of authors' names for this purpose.

%%
%% The abstract is a short summary of the work to be presented in the
%% article.
\begin{abstract}
  A clear and well-documented \LaTeX\ document is presented as an
  article formatted for publication by ACM in a conference proceedings
  or journal publication. Based on the ``acmart'' document class, this
  article presents and explains many of the common variations, as well
  as many of the formatting elements an author may use in the
  preparation of the documentation of their work.
\end{abstract}


\keywords{Do, Not, Us, This, Code, Put, the, Correct, Terms, for,
  Your, Paper}
%
%\received{20 February 2007}
%\received[revised]{12 March 2009}
%\received[accepted]{5 June 2009}


\maketitle

\section{Introduction}

Intro \dots


Exemplo de citação \cite{Abril07}.


\begin{acks}
Agradecemos aqui.
\end{acks}

\bibliographystyle{ACM-Reference-Format}
\bibliography{references}

\end{document}
\endinput
